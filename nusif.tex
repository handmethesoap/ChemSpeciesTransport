%%%%%%%%%%%%%%%%%%%%%%%%%%%%%%%%%%%%%%%%%
% Programming/Coding Assignment
% LaTeX Template
%
% This template has been downloaded from:
% http://www.latextemplates.com
%
% Original author:
% Ted Pavlic (http://www.tedpavlic.com)
%
% Note:
% The \lipsum[#] commands throughout this template generate dummy text
% to fill the template out. These commands should all be removed when 
% writing assignment content.
%
% This template uses a Perl script as an example snippet of code, most other
% languages are also usable. Configure them in the "CODE INCLUSION 
% CONFIGURATION" section.
%
%%%%%%%%%%%%%%%%%%%%%%%%%%%%%%%%%%%%%%%%%

%----------------------------------------------------------------------------------------
%	PACKAGES AND OTHER DOCUMENT CONFIGURATIONS
%----------------------------------------------------------------------------------------

\documentclass{article}

\usepackage{fancyhdr} % Required for custom headers
\usepackage{lastpage} % Required to determine the last page for the footer
\usepackage{extramarks} % Required for headers and footers
\usepackage[usenames,dvipsnames]{color} % Required for custom colors
\usepackage{graphicx} % Required to insert images
\usepackage{listings} % Required for insertion of code
\usepackage{courier} % Required for the courier font
\usepackage{lipsum} % Used for inserting dummy 'Lorem ipsum' text into the template

% Margins
\topmargin=-0.45in
\evensidemargin=0in
\oddsidemargin=0in
\textwidth=6.5in
\textheight=9.0in
\headsep=0.25in

\linespread{1.1} % Line spacing

% Set up the header and footer
\pagestyle{fancy}
\lhead{\hmwkAuthorName} % Top left header
%\chead{\hmwkClass\ (\hmwkClassInstructor\ \hmwkClassTime): \hmwkTitle} % Top center head
\rhead{\firstxmark} % Top right header
\lfoot{\lastxmark} % Bottom left footer
\cfoot{} % Bottom center footer
\rfoot{Page\ \thepage\ of\ \protect\pageref{LastPage}} % Bottom right footer
\renewcommand\headrulewidth{0.4pt} % Size of the header rule
\renewcommand\footrulewidth{0.4pt} % Size of the footer rule

\setlength\parindent{0pt} % Removes all indentation from paragraphs

%----------------------------------------------------------------------------------------
%	CODE INCLUSION CONFIGURATION
%----------------------------------------------------------------------------------------

\definecolor{MyDarkGreen}{rgb}{0.0,0.4,0.0} % This is the color used for comments
\lstloadlanguages{Perl} % Load Perl syntax for listings, for a list of other languages supported see: ftp://ftp.tex.ac.uk/tex-archive/macros/latex/contrib/listings/listings.pdf
\lstset{language=Perl, % Use Perl in this example
        frame=single, % Single frame around code
        basicstyle=\small\ttfamily, % Use small true type font
        keywordstyle=[1]\color{Blue}\bf, % Perl functions bold and blue
        keywordstyle=[2]\color{Purple}, % Perl function arguments purple
        keywordstyle=[3]\color{Blue}\underbar, % Custom functions underlined and blue
        identifierstyle=, % Nothing special about identifiers                                         
        commentstyle=\usefont{T1}{pcr}{m}{sl}\color{MyDarkGreen}\small, % Comments small dark green courier font
        stringstyle=\color{Purple}, % Strings are purple
        showstringspaces=false, % Don't put marks in string spaces
        tabsize=5, % 5 spaces per tab
        %
        % Put standard Perl functions not included in the default language here
        morekeywords={rand},
        %
        % Put Perl function parameters here
        morekeywords=[2]{on, off, interp},
        %
        % Put user defined functions here
        morekeywords=[3]{test},
       	%
        morecomment=[l][\color{Blue}]{...}, % Line continuation (...) like blue comment
        numbers=left, % Line numbers on left
        firstnumber=1, % Line numbers start with line 1
        numberstyle=\tiny\color{Blue}, % Line numbers are blue and small
        stepnumber=5 % Line numbers go in steps of 5
}

% Creates a new command to include a perl script, the first parameter is the filename of the script (without .pl), the second parameter is the caption
\newcommand{\perlscript}[2]{
\begin{itemize}
\item[]\lstinputlisting[caption=#2,label=#1]{#1.pl}
\end{itemize}
}

%----------------------------------------------------------------------------------------
%	DOCUMENT STRUCTURE COMMANDS
%	Skip this unless you know what you're doing
%----------------------------------------------------------------------------------------

% Header and footer for when a page split occurs within a problem environment
\newcommand{\enterProblemHeader}[1]{
\nobreak\extramarks{#1}{#1 continued on next page\ldots}\nobreak
\nobreak\extramarks{#1 (continued)}{#1 continued on next page\ldots}\nobreak
}

% Header and footer for when a page split occurs between problem environments
\newcommand{\exitProblemHeader}[1]{
\nobreak\extramarks{#1 (continued)}{#1 continued on next page\ldots}\nobreak
\nobreak\extramarks{#1}{}\nobreak
}

\setcounter{secnumdepth}{0} % Removes default section numbers
\newcounter{homeworkProblemCounter} % Creates a counter to keep track of the number of problems

\newcommand{\homeworkProblemName}{}
\newenvironment{homeworkProblem}[1][Problem \arabic{homeworkProblemCounter}]{ % Makes a new environment called homeworkProblem which takes 1 argument (custom name) but the default is "Problem #"
\stepcounter{homeworkProblemCounter} % Increase counter for number of problems
\renewcommand{\homeworkProblemName}{#1} % Assign \homeworkProblemName the name of the problem
\section{\homeworkProblemName} % Make a section in the document with the custom problem count
\enterProblemHeader{\homeworkProblemName} % Header and footer within the environment
}{
\exitProblemHeader{\homeworkProblemName} % Header and footer after the environment
}

\newcommand{\problemAnswer}[1]{ % Defines the problem answer command with the content as the only argument
\noindent\framebox[\columnwidth][c]{\begin{minipage}{0.98\columnwidth}#1\end{minipage}} % Makes the box around the problem answer and puts the content inside
}

\newcommand{\homeworkSectionName}{}
\newenvironment{homeworkSection}[1]{ % New environment for sections within homework problems, takes 1 argument - the name of the section
\renewcommand{\homeworkSectionName}{#1} % Assign \homeworkSectionName to the name of the section from the environment argument
\subsection{\homeworkSectionName} % Make a subsection with the custom name of the subsection
\enterProblemHeader{\homeworkProblemName\ [\homeworkSectionName]} % Header and footer within the environment
}{
\enterProblemHeader{\homeworkProblemName} % Header and footer after the environment
}

%----------------------------------------------------------------------------------------
%	NAME AND CLASS SECTION
%----------------------------------------------------------------------------------------

\newcommand{\hmwkTitle}{NuSiF project} % Assignment title
\newcommand{\hmwkDueDate}{} % Due date
\newcommand{\hmwkClass}{} % Course/class
\newcommand{\hmwkClassTime}{} % Class/lecture time
\newcommand{\hmwkClassInstructor}{} % Teacher/lecturer
\newcommand{\hmwkAuthorName}{Sam Nees, Siegfried Sch\"ofer, Paul Mildenberger} % Your name

%----------------------------------------------------------------------------------------
%	TITLE PAGE
%----------------------------------------------------------------------------------------

\title{
\vspace{2in}
%\textmd{\textbf{\hmwkClass:\ \hmwkTitle}}\\
%\normalsize\vspace{0.1in}\small{Due\ on\ \hmwkDueDate}\\
%\vspace{0.1in}\large{\textit{\hmwkClassInstructor\ \hmwkClassTime}}
\vspace{3in}
}

\author{\textbf{\hmwkAuthorName}}
\date{} % Insert date here if you want it to appear below your name

%----------------------------------------------------------------------------------------

\begin{document}

\section{Discretization of the concentration convection-diffusion equation}
	
\subsection{Fundamental equation}

\begin{equation}
\frac{\partial c}{\partial t} + \vec{u} \cdot \nabla c = \lambda \Delta c + Q(t,x,y,\vec{u},c)
\end{equation}
\begin{itemize}
	\item $\lambda$ diffusion coefficient
	\item $Q$ source term (location, time, velocity, concentration), multi-species-systems are modeled here
	\item the chemical processes under consideration have no effect on the density and hence produce no buoyancy forces $\Rightarrow$ no coupling with the momentum equations
	\item convective transport: $\vec{u} \cdot \nabla c$
	\item diffusive spreading: $\lambda \Delta c$
	\item time step control necessary
    \item for very stiff equations, implicit procedures might be necessary, we do an explicit time stepping
\end{itemize}

\subsection{Boundary conditions}
\begin{itemize}
	\item Dirichlet BC when injecting concentration
	\item Neumann BC for impermeable walls
\end{itemize}

\subsection{Discretization}
\begin{equation}
\left \lbrack \frac{\partial c}{\partial t} \right \rbrack_{ij}^{n+1} + \left
\lbrack \frac{\partial uc}{\partial x} \right \rbrack_{ij} + \left \lbrack
\frac{\partial vc}{\partial y} \right \rbrack_{ij} = \lambda \left \lbrack
\frac{\partial^2 c}{\partial x^2} + \frac{\partial^2 c}{\partial y^2} \right \rbrack_{ij} + Q_{ij}
\end{equation}
\begin{equation}
\left \lbrack \frac{\partial c}{\partial t} \right \rbrack_{ij}^{n+1} =
\frac{1}{dt} \left (c_{ij}^{n+1} - c_{ij}^n \right )
\end{equation}
\begin{equation}
\left \lbrack \frac{\partial uc}{\partial x} \right \rbrack_{ij} =
\frac{1}{dx} \left (u_{ij} \frac{c_{ij}+c_{i+1,j}}{2} - u_{i-1,j}
\frac{c_{i-1,j}+c_{ij}}{2} \right) +
\frac{\gamma}{dx} \left (|u_{ij}| \frac{c_{ij}-c_{i+1,j}}{2} - |u_{i-1,j}|
\frac{c_{i-1,j}-c_{ij}}{2} \right)
\end{equation}
\begin{equation}
\left \lbrack \frac{\partial vc}{\partial y} \right \rbrack_{ij} =
\frac{1}{dy} \left (v_{ij} \frac{c_{ij}+c_{i,j+1}}{2} - v_{i,j-1}
\frac{c_{i,j-1}+c_{ij}}{2} \right) +
\frac{\gamma}{dy} \left (|v_{ij}| \frac{c_{ij}-c_{i,j+1}}{2} - |v_{i,j-1}|
\frac{c_{i,j-1}-c_{ij}}{2} \right)
\end{equation}
\begin{equation}
\left \lbrack \frac{\partial^2 c}{\partial x^2} \right \rbrack_{ij} =
\frac{c_{i+1,j} -2 c_{ij} + c_{i-1,j}}{(dx)^2}
\end{equation}
\begin{equation}
\left \lbrack \frac{\partial^2 c}{\partial y^2} \right \rbrack_{ij} =
\frac{c_{i,j+1} -2 c_{ij} + c_{i,j-1}}{(dy)^2}
\end{equation}

\begin{itemize}
	\item concentration $c$ lies on the center of the staggered grid cell
	\item standard central differences of the Laplacian
	\item add timestep control
\end{itemize}

\subsection{Implementation}
\begin{itemize}
	\item might be possible to use an array of Arrays for multiple
			concentrations
	\item real lambda, real* c (for each specie or multiple dimensions, maybe better), real* q 
	\item extend parameter reader
	\item extend initialization
\end{itemize}
Possible loop
\begin{itemize}
	\item determineNextDT (Take the new concentration equation into account)
    \item refreshBoundaries: Dirichlet and Neumann BC according to input for every species (e.g. iterate over all arrays)
    \item computeFG: no change
    \item composeRHS: no change
    \item calculateConcentrations: update the concentration in an explicit way
    \item updateVelocities: no change
    \item vtk: no change 
\end{itemize}

\subsection{Application}
\begin{itemize}
    \item catalysator (3 species)
	\item windtunnel
	\item mixing of two colours
\end{itemize}

\end{document}
